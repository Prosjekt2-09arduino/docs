\chapter{Introduction}

This document is the report required by the course IT2901 at the Norwegian University of Science and Technology (NTNU), written during spring 2013. The goal of this report is to give the reader an overview of the project through documentation, organization and management.
The task was to develope an market application on Android for Arduino programs, and installing those applications over bluetooth.

\section{The Course}
IT2901 Informatikk Prosjektarbeid II is the course in which this project was done through. It is commonly taken the last semester of a bachelor in informatics. This report is written for the customer, supervisor and teachers of this course and thus may be difficult to understand for others.

\section{The group}
The group consisted of six students from Computer Science at NTNU, each possessing different technical skills, but all were third year students in the Informatics Bachelor programme.

\begin{description}
	\item{\textbf{Bjørn Arve Fossum}}\hfill \\
		Experience with Java, MySQL, PHP and C\#.
	\item{\textbf{Jeppe Eriksen}}\hfill \\
		Experience with Java, MySQL, and basic Android development.
	\item{\textbf{Nina Margrethe Smørsgård}}\hfill \\
		Experience with Java, MySQL, \LaTeX, and basics Android development.
	\item{\textbf{Robin Tordly}}\hfill \\
		Experience with Java, Android, MySQL, PHP and SQLite.
	\item{\textbf{Ståle Semb Hauknes}}\hfill \\
		Experience with Java, MySQL, PHP and Arduino$\texttrademark$.
	\item{\textbf{Wilhelm Walberg Schive}}\hfill \\
		Experience with Java, MySQL, and the basics of Arduino$\texttrademark$ project development.

\end{description}

\section{The Customer}
SINTEF is the largest independent research organisation in Scandinavia. It is an independent, non-commercial organisation with close ties to NTNU and international activity with clients in 60 different countries. There is a joint effort by NTNU and SINTEF regarding lab work, many are employed by both and there are labs that are funded by both companies in order for continued research.

\section{Problem description}
The project task was to develop a platform for easy use, setup and sharing of Physical User Interfaces (PUIs) for Arduino$\texttrademark$. The task was divided in three main parts: a market application, over the air installation and example PUIs.\\
\newline
The purpose of the market application was to allow users of Arduino$\texttrademark$ to browse and download PUI applications for their Arduino$\texttrademark$ board on a mobile Android device. This market app was to be a simplified version of e.g. Google Play, where users easily can browse and install whatever applications they desire. An application selected for installation in the market application should be prepared on the mobile device, and installed over the air on an Arduino$\texttrademark$ board using Bluetooth$\textsuperscript{\textregistered}$ . The example PUIs were mostly intended to demonstrate the feasibility of the finished product.

\section{The goal}
The goal of the project task was to make Arduino$\texttrademark$ easier to use for ordinary people, by allowing easy browsing, sharing and installation of applications on Arduino$\texttrademark$ boards. By developing an application for over the air installation of applications, the finished product should ease the process of both installing and updating PUIs on an Arduino$\texttrademark$.


\section{Definitions}
This is a list of terms and abbreviations used throughout the project report in order to clarify and explain their meaning.

\begin{description}

\item[Android:]\hfill \\
An operating system for mobile devices based on the Linux operating system. It is developed by Google and the Open Handset Alliance. Applications for Android devices are written in Java, and all the software is open source released under the Apache License.

\item[Apache:] \hfill \\
A software foundation focused on open source and community driven software.

\item[Arduino$\texttrademark$:]\hfill \\
A tool for making Physical User Interfaces (PUIs). It is an open-source physical computing platform based on a simple microcontroller board.

\item[AVRDude:]\hfill \\
This is a tool to upload programs to microcontrollers from the Unix command line. This software has a lot of features, such as reading and writing to the microcontroller's memory over a serial connection. This software can also compile your C code into an Intel Hex file, a file full of binary information, that can be sent to the Arduino$\texttrademark$ with the STK500 protocol. Normally this is done with a cable between a computer and the microcontroller.

\item[STK500:]\hfill \\
This is a standard protocol used by many microcontrollers to upload and download memory, including the Arduino$\texttrademark$.

\item[PUI:]\hfill \\
An acronym for Physical User Interface. A PUI is a user interface which interacts with digital information through the physical environment.

\item[JUnit:]\hfill \\
A unit testing framework specifically written for Java.

\end{description}
