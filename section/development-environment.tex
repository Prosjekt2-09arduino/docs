\chapter{Development environment}
In this chapter the different tools and resources used in relation with the project will be presented. A general requirement for a tool to be used, was that it was multi-platform, as three different operating systems were in use within the group. It was also preferable for the tool to be free, since this is a project that does not generate income.

\section{Development tools}
These are tools used in the development of the project. They have contributed in the technical aspects of development such as coding.
\subsection{Integrated development environments (IDEs)}
An IDE is used by programmers to develop and program applications. These are the IDEs that was used during the project.

\subsubsection{Eclipse}
Eclipse is a freely available IDE implemented in Java. It has extensive plugin support and anyone can publish a plugin to support another programming language, versioning software or new program features altogether. Eclipse provides useful, time saving functionality to developers, such as an extensive live debugging suite in addition to checking code syntax and providing auto completion of method calls.\\
\newline
Eclipse was chosen for Android development, due to previous experience with the software and good plugin support for targeting different versions of the Android API. Using a plugin for Arduino support was also considered, but the functionality of the available plugin was found lacking. % TODO Ståle, kan du si mer her?

\paragraph{Mylyn}
is a plugin meant to integrate with other plugins that provide access to task, issue and bug tracking repositories. Issues can be referenced or created quickly from the Eclipse IDE while coding, and the plugin can be configure to show relevant code sections when selecting an issue to work on. The group used a GitHub connector for Mylyn to get Mylyn to display and work with GitHub issues.

\paragraph{The Android software development kit (SDK)}
provides access to several versions of the Android API through the use of an installation manager. This allows for simple updating when new versions of the API are released, as well as supporting older devices. A simulator for testing against different Android devices is also available.

\subsubsection{Arduino IDE}
The Arduino IDE is provided by the creators of the Arduino platform as a free and open source program. It provides syntax checking, example programs, as well as basic editing tools. The IDE handles compilation of Arduino code into C and C++ code the regular micro-controller tools can handle (the IDE includes and depends on tools developed by Atmel, the company behind the chip Arduino uses) and can pass it on to an integrated uploading tool (avr-dude) to get it running on the Arduino.\\
\newline
The official Arduino IDE was considered for coding the Arduino component in, but due to its limited text and code related feature set (lacking features such as auto complete), the IDE was not used for coding.

\subsection{Codebase management and versioning}
The tools described in this section were used for versioning and managing the codebase.

\subsubsection{Git version control software}
Git is a free, open source Version Control Software (VCS) with support for both local and remote repositories. It keeps track of differences between versions of files and allows for offline commits and branch creation, as this is done locally first before pushing the updates to a repository.\\
\newline
There are various terminal and front-end solutions available, including a GUI version from GitHub and Eclipse plugins. Past experiences with Git plugins for Eclipse led to a decision to use terminal software, as the GitHub program lacked advanced functionality regarding branching, among others. Another benefit with that solution was that the VCS then was IDE- and platform-independent and that there was no need to use the IDE for working with the report repository.

\subsubsection{GitHub}
\label{sec:GitHub}
GitHub hosts repositories for use with the Git VCS. They also provide basic issue tracking and social features, such as following other developers or projects. Paying customers can elect to hide their code, while free users have to share their code with everyone (one user can however keep one hidden project as long as he is the sole developer). The customer required source code to be uploaded to GitHub and also recommended use of GitHub for requesting assistance using issues - a recommendation that was followed.

\section{Project management tools}
The tools mentioned here were tools used in general management of the project. This includes elements such as the planning of meetings, writing and sharing information between group members.

\subsection{Google Docs}
Google Docs is a free to use online office suite that allows users to simultaneously edit documents of different types (text, spreadsheet, presentation, et cetera), which was of great use when the group was working together on documents. It also supports document chat and comments (temporary chat/forum thread-like structure). Google Docs was used for editing the preliminary version of the report and other documents concerning the project.

\subsection{Microsoft Word}
Microsoft Word is the de facto standard Word Processor, with support for many different effects (blinking letters, 3D text and so on). Content creation and formatting tasks are nearly completely intertwined, making it tricky to maintain a consistent document as the documents grow larger. Microsoft Word was used to generate certain graphs for the preliminary reports.

\subsection{\LaTeX}
\LaTeX is a macro-improved version of Tex used for typesetting documents. It is free and open source. The general idea is to separate content creation and formatting to create consistent documents effectively without being distracted by the appearance of the content. After the preliminary report, work on the report was written in \LaTeX.

\subsection{Google Calendar}
Google Calendar is a free to use online calendar. It supports both private and public events. Google Calendar allows for simple calendar synchronization across different devices and platforms. A shared calendar in Google Calendar was created to make it easier to keep track of group meetings, meetings with the customer and so on.

\subsection{Dropbox}
Dropbox is a cross platform file synchronization and sharing service. Files in a specified folder are automatically kept in sync among different devices. It is also possible to share folders with other users and with the world at large (this will also serve HTML pages). The basic service is free, but comes with limited storage space. Dropbox was used for sharing of various files within the group, such as compiled versions of the report, data sheets for Arduino components and so on.

\subsection{TextMate and Sublime Text 2}
These tools are sophisticated text editors that can be used for coding, markup and regular text editing. In the project they were used for editing of the report and coding for Arduino. While TextMate is for OS X only, Sublime Text is cross-platform.

\subsection{Wunderlist}
Wunderlist is a free todo list tool supporting shared lists. It supports sharing tasks between the participants of a to-do list with reminders and notes. Wunderlist was used to serve reminders for other, non-coding related tasks, like scheduling meetings or booking rooms.

\subsection{Doodle}
Doodle is a free online tool for scheduling events. It allows a user to create an event and aims to simplify scheduling of events by `polling" the participants when they are available. When the participants of the event have answered, one can easily view when all participants are available for meeting. Doodle was used to schedule the first meetings the group had, before regular meetings were established.

\subsection{Fritzing}
Fritzing is a tool that lets you draw circuit boards and wiring diagrams. This makes simple diagrams that are easy to understand and easy to reproduce for users that do not understand electronics. Often used with the Arduino platform because usually the users of this platform have no electronic experience. Most of the standard electronic equipment used with the Arduino are included in this tool. Fritzing was used to create the wiring diagram of the Bluetooth wiring and the iJacket clone.

\subsection{OmniGraffle}
OmniGraffle is a software used to draw diagrams and charts. This program makes it easy to draw neat diagrams and charts without spending much time on this. OmniGraffle was used to create ER-diagram and use cases.

\section{Test management}
The tools mentioned here were used for preparation and creation of tests for the project.

\subsection{Robotium}
Robotium is a JUnit based framework for the creation of automatic unit testing in Android. It automates the created tests by going through the Android application as described in the tests. This shows the tester what is done and even though it may be slow, it allows the tester to see if a mishap is due to the test itself being wrong, for example being in the wrong screen.
