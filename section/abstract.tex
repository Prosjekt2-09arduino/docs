\begin{abstract}

In cooperation with IDI at NTNU, SINTEF was the contracting customer for this project, hiring the group consisting of six students on the course IT2901.\\
\\
The project's aim was to expand the usability of existing Arduino devices, and simplifying the process of installing new functionality.
This report is an overview of the project of developing the system to do so, and how the work of the developer team evolved alongside the demands and needs of the customer.
The base idea was to create an easy-operable system running on Android based devices for installing compatible applications to the customers' existing Arduino device.
This was done by implementing the STK-500v1 protocol for AVR061 onto the Arduino Uno board, and creating a market-application for Android that handles connection via Bluetooth and data transfer over that protocol, as well as containing the market for browsing of available apps.\\
\\
The development of both the protocol and the app was done in Java, as both were designed to run on Android devices. This was a small challenge as the protocol originally was written in C for the AVR061. Loading of addresses while writing to the Arduino proved to be the thing delaying integrating development of the app and the protocol into one package, requiring modification of requirements.\\
\\
The Arduino side of the software was written in C, but much could be reused from a prior project.

\end{abstract}
