\section{Prestudy}
\subsection{Development methods}
\subsubsection{Scrum}
Scrum is an iterative and incremental development framework for development of complex information systems. When using Scrum, the development of the product is developed in small pieces, with each piece building upon previously created pieces. The use of Scrum requires the team to be divided into specific roles, where each role has its own responsibility. The following is the core roles of the Scrum team:
\begin{description}
	\item[Product owner:]{The product owner represents the stakeholders in the project, and is the voice of the customer. It is the product owner's responsibility to ensure that the Scrum team at all times is working with the right things seen from a business perspective.}
	\item[Scrum master:]{The scrum master should act as a buffer between the development team and distracting influences, so that the development team can deliver potentially shippable products at the end of each sprint. The scrum master should keep the development team focused at all times.}
	\item[Development team:]{The development team is made up from three to nine persons with cross-functional skills. This team does the all the actual work, including development, testing, designing and so on.}
\end{description}

\subsubsection{Waterfall model}
The waterfall model is among the first process models to be introduced. In this model, each phase in the development must be completed before one can advance to the next phase. At the end of each phase in the development, a review takes place. In this review it is determined if the project is on the right path and if it should continue or not. The phases in the development does not overlap.

\subsubsection{Spiral model}
The spiral model is software development model intended for large and complicated projects. It combines elements from the waterfall model and prototyping models, and uses an iterative approach. Based on this it allows for incremental releases of the product, as prototypes.

%Check if it is written anything about this in the Systemutviklings-bok.
\subsubsection{Lean software development}
Lean is an agile software development methodology that is defined by seven principles:
\begin{description}
	\item[Eliminate waste.]{Everything that does not add any value to the customer is considered a waste.}
	\item[Amplify learning.]{The development environment should the designed so it amplifies the learning process of the developers. Defects should be prevented by running tests as soon as the code is written.}
	\item[Decide as late as possible.]{By delaying decisions as much as possible, better results can be achieved. There is always uncertainty associated with software development, and delaying options as much as possible results in more flexibility later.} 
	\item[Deliver as fast as possible.]{The sooner the product is delivered, the sooner feedback can be received.}
	\item[Empower the team.]{The managers are taught how to listen to the developers, so they can explain better what action that might be taken, and give suggestion for improvement.}
	\item[Build integrity.]{The customer needs to have influence and inspection of the project, and an overall experience of the system to be developed.}
	\item[See the whole.]{Decompose the system into smaller parts and find and eliminate defects.}
\end{description}

\subsection{Existing solutions}
This is a summary of existing solutions similar to the project assignment. This section is divided in two subsections; one for the market application and one for the over-the-air transfer. The existing products were evaluated on the following criteria:
\begin{itemize}
	\item{To what degree the product fits the assignment.}
	\item{Can the product, or parts of it, be reused for the assignment? Can it lead to licensing issues?}
\end{itemize}

\subsubsection{Market applications}

{\bf UbiBazaar} UbiBazaar is a prior project which created an universal application store in PHP. This project had potential for serving as a back end, but this was later discarded. The idea of a website for uploading applications was the primary idea for the project. Upon revising the project it was deemed too unstable and incomplete to properly function for our intended use. The project used an outdated version of PHP which caused numerous bugs which would take up too much time to attempt to fix properly.\\

It did have a way for developer to follow each other, which would help creating a community for the developers. For further development of µCSS, UbiBazaar would also be able to support the development of applications aimed at, for example, raspberry pi. This was not included as the goal of our project and therefore discarded.

\\

{\bf Google Play} is the market application for Android provided by Google. It is an application that allows for easy browsing and installation of applications on Android devices. It also detects what model of phone that is being used, and only shows applications that are supported by that phone. This functionality is similar to the requirements for the Android application in the project.

\\

\begin{figure}[H]
\includegraphics[scale=0.2]{images/Google-Play-Store-APK-3-7-15.jpg}
\caption{Google Play Store}
\end{figure}

\\

Google Play fits the assignment in the way that is is a market applications where one can download applications. This was useful for the development of the product, since it was possible to use the same principles in the assignment. It was also similar in the way that it was possible to browse for applications on the computer, and ''push'' the application to a mobile telephone. This, however, did not connect via bluetooth which the task assignment stated that the finished product should. Google Play is not open source and could therefore only be used as a source of inspiration for the project. It was not possible to reuse the code or other parts of the application in the development. 

\subsubsection{Over the air transfer}
Pebble is a watch that offers over the air transfer of applications. It is based on the same microchip as one of the newest Arduino, but contains an operating system written in C.
\begin{figure}[H]
\includegraphics[scale=0.7]{images/Pebble-Smartphone-Watch.jpeg}
\caption{Pebble Watch}
\end{figure}
There was very little documentation on the Pebble website, so the potential reuse in this project appears minimal.