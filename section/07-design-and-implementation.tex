\section{Design and implentation}
The market app should be easy to use and fit for consumers of all ages. To install an application the user has to select a desired application with one touch, then check the specification if its compatible with the Arduino, then a agree/install button for installing the app on the Arduino. This is a total of two clicks, despite you have chosen a category or search hits on the market app. This is similar to Google Play.\\
\newline
Each (supported) Arduino device is equipped with a Unified Resource Identifier (URI) stored in its ROM. This URI is automatically transferred to the market app once it is turned on and is in range; the hardware configuration (Memory, connected devices) is stored in a XML file the URI links to.
The app store client can then compare compatibility with the device and app requirements.
Only apps supported on your Arduino will initially be visible in the market application, however, power users can enable unfiltered results to browse the selection and consider upgrading the device.\\
\newline
Anyone modifying a board will have to provide a hardware specification and transfer a new URI to the device if they want filtered results in the app store (Such a user could still choose to view and install unfiltered applications).
The URI could also be read from a QR-code.