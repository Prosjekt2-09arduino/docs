\chapter{Evaluation}

This chapter covers the evaluation of the final product, the process model and what the team learned in this course.

\section{Problems and difficulties}
	There were problems that arose during the course of this project. This section focuses on the two biggest of these problems, licensing and the STK500 protocol.

	\subsection{Licensing}
	No member of the group had any experience with licensing before this project. When the customer then told us to freely copy open-source material for use in the project. This project in its entirety is licensed with Apache 2, meaning that systems licensed under GPL 1 or 2 could not be used. Thus instead of using these systems, or parts of them, they had to be written from scratch.

	\subsection{Protocol}\label{sec:protocol-issues}
	There were several difficulties related to programming the Arduino. There were few existing solutions for programming it in general, and none for Android or Java. The official implementation was available in C, but in the incompatible GPL license.\\

	The existence of certain programs written in Java with the purpose of programming Arduinos,
	led the group into assuming one of these could be utilized, as the license was compatible. When the time came to make use of it, it was discovered to internally use AVRDUDE as well - an embedded version was available for several platforms.\\

	This realization came somewhat late in the process, see Iteration 4 in chapter \ref{sec:Iteration4}. Following this and the emergency meeting with the customer, it was decided to implement the protocol for programming Arduinos. Unfortunately, the documentation for these were a bit confusing; the most detailed and professional protocol document turned out to describe version 2 of the protocol\cite{AVR068}, which was incompatible with the bootloader installed on the Arduino by default (Optiboot v4.4 on Arduino Uno). This was also the version that was partially implemented before the protocol differences and lack of backwards compatibility was discovered.\\

	The incompatibility could have been discovered earlier, but testing of the protocol had to be postponed due to a lack of available working serial communication libraries for PC's compatible with the Arduino Uno USB interface, so a simple Android I/O app was added for testing purposes. Even then, the reason for the failure to communicate with the Arduino was hard to pin down, until a second Arduino was wired up to echo back the information the pair received.\\

	Changing the bootloader to utilize version 2 of the protocol could not be done due to it being licensed under GPL. Some attempts were made at finding compiled bootloaders for the Uno, but did not bear fruit. Attempts at compiling bootloaders using version 2 for the chip failed due to unresolved dependencies in the bootloader projects discovered, like the Stk-Boot\cite{StkBoot} and uOS-Embedded\cite{uOS-Embedded}.\\

	The solution, then, was to implement version 2 of the protocol, and write off most of the protocol work as waste (a bit less than a week); some of it could be reused despite the protocols being radically different, however. The second version of the protocol has a standardized message format with checksums and static tokens inside the messages to aid in error discovery; this would have been beneficial for communicating over Bluetooth.\\

	It was known that Optiboot didn't make use of all the commands specified in the protocol document AVR061\cite{AVR061}, but documentation on which commands were required or superfluous was lackluster. Since it would simply report that everything is fine when encountering such a command, discovering which were implemented or not had to be done by trial and error or investigating the source (which is designed to be very compact, to fit in the 512kB threshold, which, if exceeded, would make the bootloader occupy 1024kB).\\

	Communicating with the Arduino proved fairly straight forward, but dealing with false positives was an issue.
	The Arduino would report that writing took place successfully, but writing would occur only in partial sections of the memory areas requested, for example. The only way erroneous writing could be discovered, was to read afterwards to verify the contents (or observe the Arduino not starting run the code it was supposedly programmed with).\\

	These issues prolonged the development of the system by a considerable amount, and had a major impact on the project. Some
	of these could have been avoided by more careful initial investigation, while others would not have been apparent prior to implementing the protocol.

	\subsection{Hard reset}


\section{Interaction with the customer}

	The communication with the customer was great. A regular meeting was held every Friday, and the customer had an insight to the projects repositories, this way the customer always knew what was going on. The customer was also very forthcoming and helpful. This is illustrated by the fact that when the group encountered the problem with the protocol he said he was able to request for third party help with the problem in order to quicken the task at hand as time was becoming a big issue. 



\section{Group interaction}
	The group decided upon regular meetings three days a week, where people sat down and worked together, discussed issues and progress. It was said that during a larger course like this, every member should work approximately 20 hours every week, thus this was quickly agreed upon. Apart from the regular meetings, the members were able to work at flexible hours within the iterations. There were never any issues within the group, causing trouble, slowed work or other things. All in all the members were happy with the given group and work was steady all through the semester.\\ 
	In Table~\ref{table:workhours} the total workhours for the group on every iteration is listed. The reduction in iteration 4 was due to the Easter holidays.

	\begin{table}[H]
	\caption{Iterations and time spent}
	\centering
	\label{table:workhours}
	\begin{tabular}{|l|l|}
		\hline
			{\bf Iteration} & {\bf Hours}\\
		\hline
			Iteration 1 (Week 6-7) & 186 hours\\
		\hline
			Iteration 2 (Week 8-9) & 257 hours\\
		\hline
			Iteration 3 (Week 10-11) & 233 hours\\
		\hline
			Iteration 4 (Week 12-13) & 129 hours\\
		\hline
			Iteration 5 (Week 14-15) & 195 hours\\
		\hline
			Iteration 6 (Week 16-17) & 248.5 hours\\
		\hline
			Iteration 7 (Week 18-19) & 263 hours\\
		\hline
	\end{tabular}
	\end{table}

	An associated chart diagram of the workhours are shown in Figure~\ref{fig:workhours}

	\begin{figure}[H]
	\centering
	\captionof{figure}{Workhours in each iteration throughout this project}
	\label{fig:workhours}
	\includegraphics[scale=0.8]{images/workhours_chart2.png}
	\end{figure}



	\section{Process}
	The overall process went well. The group has learned to document issues and tasks underway. This made the process flow steadily forward, and made it clear and organized. Every member always knew what to do when he or she checked the issue list on GitHub.

	The communication within the group was also good, the group worked together almost every day and had a group meeting every week. The group members could also contact each other via Skype, phone and email.

	Mistakes were made however. During the first few weeks of the project, much of the focus was on the Android application; be it design or actual coding for Android. The general consensus was that the Android application would take approximately the same amount of time as the protocol for transferring files from the Android device to the Arduino device. This was on the basis that a Java implementation of the STK500 protocol existed, but poor research was done and it proved to be unusable (see chapter \ref{sec:protocol-issues} for details).


	\section{Conclusion}
    The project was successful and the most important requirements including the biggest problem (the protocol) was met. The frequent group meetings, customer meetings and the organized structure of issues, turned out to be a great success. \\

    	\subsection{Requirements met}
    		The graphical user interface was completed and met every requirement.\\

		\begin{itemize}
			\item{GUI}
			\item{Bluetooth connection}
			\item{QR Code and SerialInput}
			\item{Retain a reliable connection to the connected device}
			\item{Remember the device from last time and try to reconnect}
		\end{itemize}
    		Bluetooth connection was complete and met every requirement. This part of the project was taken from the UbiCollab/iJacket project and spared the group a lot of workhours. The disadvantage of this usage, was that the bluetooth connection protocol turned out to be unstable.\\

    		The system will automatically try to connect to the last sucessfully connected device on startup, and gives the user the option to turn this feature on and off. This works perfectly if the content on the Arduino is correct. This requirement is considered as fully met.\\

		\subsection{Requirements not met}
			Content provider with sync adapter towards a remote database was not implemented. This was not prioritized because it has been done by many other projects and this was agreed upon with the customer. \\

			Filtering applications by compatibility was not implemented because of the lack of time the project had. It was a complex problem that was considered to take a period of at least one month past the project schedule. Nevertheless the team managed to write a standard for microcontrollers, that describes how this could be implemented. A document of this standardization can be found in appendix C. \\
