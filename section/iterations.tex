\chapter{Iterations}

This section contains an overview and a summary of each iteration of this project. The project was split into
iterations of two weeks, and the content is based on the activity plan and the status report that was delivered to the supervisor. In each iteration the use cases described in chapter ~\ref{usecases} is shown to present what the group was working on.

\section{Iteration 1}
Week 6 and 7
\subsection{Summary}
	These were the first two weeks of the project and the first group meeting was arranged.	Meeting with the customer to get the requirement specification and understanding of the projects goal was done.	Required libraries, licenses and other factors was taken in account before the group could start developing. No use cases were generated in this iteration, as basic research and planning had to be done before scenarios could be generated.
\subsection{Overview}
\begin{itemize}
	\item{Confirm understanding of the task}
	\item{Ideas for identifying the capabilities of an Arduino device and how to communicate these to the store client}
	\item{Investigate potential solutions to facilitate over the air installation of Arduino applications}
	\item{Start design the GUI for the Android application}
	\item{Research on Apache licenses}
	\item{Meeting with customer}
\end{itemize}

\section{Iteration 2}
Week 8 and 9
\subsection{Summary}
	Architectural work and design of the user interface of the Android application. Positive feedback on the GUI was received from the customer. Design of use cases and further research was performed.
\subsection{Overview}
\begin{itemize}
	\item{Think architectural}
	\item{Designing of GUI}
	\item{Research on over the air implementation}
	\item{Research Ubicollab libraries}
	\item{Use iJacket - generic application, Bluetooth connection code}
	\item{Implement a first sketch of the content provider}
	\item{Meeting with customer}
\end{itemize}

\section{Iteration 3}
Week 10 and 11
\subsection{Summary}
	Focus on Bluetooth connection and over the air implementation was prioritized these weeks. The application was taking form and the bugs were beginning to show up. Use cases 2, 3, 4 and 5 were in focus during this iteration. There were some issues with over the air because of incompatible licenses and much work was put into research. The Android application was estimated to be 50\% complete.

\subsection{Overview}
\begin{itemize}
	\item{Research Ubicollab libraries}
	\item{Finish design of the GUI}
	\item{Establish a Bluetooth connection between Android and Arduino}
	\item{Implement last draft of the content provider}
	\item{Meeting with customer}
\end{itemize}

\section{Iteration 4}\label{Iteration4}
Week 12 and 13
\subsection{Summary}
	This iteration became critical when it was discovered that existing implementations of the STK500 protocol were either incompatible with the required Apache v.2 license or not compatible with Android. An emergency meeting with the customer took place and it was decided to implement the STK500 protocol in Java. Developing the Sync Adapter was removed from the requirements, and the focus was moved to implementing the protocol as a library. Use cases 2, 3 and 4 were in focus during this iteration.

\subsection{Overview}
\begin{itemize}
	\item{Emergency meeting with customer}
	\begin{itemize}
		\item{Not possible to use existing solutions of STK500 protocol}
		\item{Agreement on implementing the STK500 in Java ourselves as a library}
	\end{itemize}
	\item{Research on STK500}
	\item{Unit test sketches}
	\item{Continued work on develop the Android application}
\end{itemize}

\section{Iteration 5}
Week 14 and 15
\subsection{Summary}
	Good progress and much was done. The group was more optimistic about the implementation of the protocol and the development of the application was almost complete.	The group was now split into two, where one group was writing the protocol, and the other group was writing Unit tests and polishing the last parts of the application. Use cases 4, 5 and 7 were in focus during this iteration.

\subsection{Overview}
\begin{itemize}
	\item{Added storage of hex-files to content provider}
	\item{Writing Robotium unit tests}
	\item{Writing the STK500 protocol in Java/Android}
	\item{Finishing the Android application except for the STK500 part}
	\item{Meeting with customer}
\end{itemize}

\section{Iteration 6}\label{iteration6}
Week 16 and 17

\subsection{Summary}
	Unexpected issues with programming the Arduino nearly caused the implementation to fail, requiring a meeting with the customer.
Suggestions for alternate solutions were presented, and it was decided to work in parallel on some of them. The customer was to attempt to ask some experts in the field for some code review.
Shortly after the meeting the problem was resolved.

\subsection{Overview}
\begin{itemize}
    \item{Meeting on protocol implementation problems}
    \item{Problem resolution}
	\item{Implement the STK500 to the Android application}
\end{itemize}

\section{Iteration 7}
Week 18 and 19

\subsection{Summary}
These were the last two weeks scheduled for the project.\\
A new wrapper was created to solve the problem encountered in \ref{iteration6}.
The final parts of the application were completed and the STK500 was implemented into the application as a library.
	A workshop with the customer at SINTEF was held, and an acceptance test and a usability test were performed. Most of the time was spent on completing the documentation.

\subsection{Overview}
\begin{itemize}
	\item{Workshop for SINTEF}
		\begin{itemize}
			\item{Demonstrate our finish application}
			\item{Acceptance testing}
			\item{Usability testing}
		\end{itemize}
\end{itemize}
