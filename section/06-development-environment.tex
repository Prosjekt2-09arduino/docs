\section{Development environment}
\subsection{Development tools}
\subsubsection{Integradet Development Environments (IDEs)}
Eclipse was chosen for Android development, due to previous experience. The official Arduino IDE was considered, but due to its limited feature set (lacking features such as auto complete) it was decided to use Eclipse for this as well.\\
Mylyn plugins were used to integrate GitHub issues into the IDE.

\subsubsection{Codebase management and versioning}
The customer required source code to be uploaded to Github. Use of git terminal solutions was deemed useful, since these are similar across different platforms and are IDE-independent.

\subsection{Communication}
Most of the communication within the group were done at group meetings and when the group were working together. For communication between group members outside the meetings it was decided that the group should only use email and Skype. This was decided to avoid the confusion that might arise from using numerous channels of communication. Mobile phone was also used when immediate contact were necessary.\\
\newline
Communication between the group and the customer were mainly done in meetings or by email. The same applies for communication with the supervisor.\\
The customer heavily recommended use of Github for requesting assistance using issues and milestones, which was accepted for use.\\
Wunderlist was used for other, non-coding related tasks.

\subsection{Project management tools}
Google docs was used for editing of the report and other documents in connection with the project. It allows for simultaneous editing of documents which were of great use when the group was working together.\\
The final version of the report was created using \LaTeX.
A shared calendar in Google Calendar was created to make it easier to keep track of group meetings, meetings with the customer and so on.\\
Dropbox was used for sharing of various files within the group. 
Microsoft Word was used to generate certain graphs for the preliminary reports.\\
\newline
Scrumy.com is a online project management tool loosely based off scrum, and was used as the scrum task board during the project. Since the group did not have a constant meeting area, it was helpful to have the scrum task board online instead of using a physical board.