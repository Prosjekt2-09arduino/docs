\chapter{Project Management}
The following sections details the project management decisions of the project. This includes choice of development method, team member responsibilities, communication channels and risk analysis.

\section{Project planning}
\begin{figure}[H]
\includegraphics[scale=0.8]{images/gantt-diagram.png}
\caption{Gantt diagram. Milestones are marked using vertical lines}
\end{figure}

\section{Development method}
The Rational Unified Process (RUP) model is an extensible framework\cite{kruchten} and was chosen for the project. By this we mean that it should be customized for a specific project.
It is an iterative and incremental software development process that is divided into a series of timeboxed iterations.

Each iteration results in a increment, which is a release of the system that is and an improvement of the previous iteration.
Most of the iterations will contain processwork as requirements, design, implementation and testing.

A feature of this model is that it is use case driven. Every iteration takes a set of use case scenarios from the requirements and use those for the content of the iteration.

The Unified Process model requires the team to focus on the critical risks of the project early in the development phase. This is especially in the Elaboration phase, where the greatest risks are addressed first.

The model consists of four phases:

\begin{itemize}
\item{Inception}
\item{Elaboration}
\item{Construction}
\item{Transition}
\end{itemize}

The Inception phase is the smallest phase and should contain the work of identifying risks and set the portion of requirements and use cases to work on in this iteration.

In the Elaboration phase the team is expected to filter out the majority of the system requirements and validate the system architecture.At the end of the Elaboration phase there should be created a plan and schedule estimates for the Construction phase. This will be the activity plan and the status report that is used in this project.

The Construction phase is the largest phase where all the features is implemented into the system and the textual use cases is made.

The last phase: Transition phase, is where the system is deployed to the target users. Feedback might result in further refinements. This step could also include user training which was decided to not be done. This was mostly because the eventual software product would not be an end user product, but a prototype, and in part because the system should be user friendly and easy to understand.

\subsection{Changes to description}
The development method description was refined between the preliminary version and the midterm version of the report. Based on the customer requirements of short iterations, it was clear that an iterative model was in order. Use of Lean principles was adopted from the start of the project as well.\\
The specific documentation of the development method was deferred for three weeks, however. This was due to the uncertain future of the project (over the air installation was said to potentially be impossible) and the need to get the first iteration presented to the customer; precisely documenting the development method was deemed to be wasteful at that point in time.

\section{Team roles}
The group was organized in different roles based on skill and experience. Each team member was given a responsibility for some code-packages. Further, the team was divided in six subgroups where each subgroup had one responsible leader. These were respectively group leader, documentation and substitute leader, Android and GUI, Arduino$\texttrademark$ and PUI, over the air and test leader. Work was done in subgroups of two, which made it easy to do both pair-programming and individual work.

\subsection{Role evaluation}
The division of the group was an important feature. Every member knew who to contact about a specific problem or task.\\

\begin{description}
	\item[Group leader]{was responsible for the progress in the overall project. This person ensured progress and priorities for deadlines.}
	\item[Documentation and substitute leader]{was responsible for management of documentation and reports. In absence of the group leader, this person took on the group leader's responsibilities. This person was also responsible for contact with the customer and supervisor.}
	\item[Android and GUI]{was responsible for the Android part of the project.}
	\item[Arduino$\texttrademark$ and PUI]{was responsible for the Arduino$\texttrademark$ part of the project. This implies contacting the Arduino$\texttrademark$-lab, requisitions for hardware, the coding part and over-the-air installation. This role was also responsible for development of the PUI examples.}
	\item[Over the air leader]{was responsible for programming the Arduino$\texttrademark$ over the air. This person was also responsible for making the first prototype with a Bluetooth$\textsuperscript{\textregistered}$  connection.}
	\item[Test leader]{was responsible for developing and executing tests for the complete project.}
\end{description}

\begin{table}
\begin{tabular}{|l|l|}
\hline
	{\bf Name} & {\bf Role}\\
\hline
	Jeppe Benterud Eriksen & Group leader\\
\hline
	Nina Margrethe Smørsgård & Documentation and substitute leader\\
\hline
	Robin Tordly & Android and GUI leader\\
\hline
	Bjørn Arve Fossum & Arduino$\texttrademark$ and PUI leader\\
\hline
	Ståle Semb Hauknes & Over the air leader\\
\hline
	Wilhelm Walberg Schive & Test Leader\\
\hline
\end{tabular}
\caption{Roles}
\end{table}

\section{Communication}
Most of the communication within the group was done at group meetings and when the group was working together. For communication between group members outside the meetings it was decided that the group should only use email and Skype. This was decided to avoid the confusion that might arise from using numerous channels of communication. Mobile phone was also used when immediate contact was necessary.\\
\newline
% What about GitHub?
Communication between the group and the customer was mainly done in meetings or by email. The same applies for communication with the supervisor.

\section{Risk analysis}
\captionof{table}{Risk analysis}
\label{fig:risktable}
\begin{table}[H]
\begin{tabularx}\linewidth{|m{0.15 \textwidth}|m{0.1 \textwidth}|m{0.1 \textwidth}|m{0.1 \textwidth}|X|X|}
\hline
	\textbf{Description} & \textbf{Likeli{-}hood} & \textbf{Impact} & \textbf{Impor{-}tance} & \textbf{Preventive Action} & \textbf{Remedial Action}\\
\hline
	Illness & 7 & 2 & 14 & Good communication and effective use of GitHub & Increase workhours and exchange tasks and responsibilities\\
\hline
	Project\newline complexity & 6 & 5 & 30 & Don't take on too much work & Cut down the demands\\
\hline
	Customer\newline issues & 1 & 5 & 5 & Agreement with customer and weekly feedback from customer & Use the original requirement specification\\
\hline
	License\newline incompability & 7 & 7 & 49 & Acoid integrating components with incopatible licenses & Discover other implementations or implment from scratch\\
\hline
	Group\newline conflicts or disagreements & 3 & 3 & 9 & Keep close contact to avoid surprises. Leader takes action & Contact supervisor and make an appointment\\
\hline
	Over the air complexity & 8 & 8 & 64 & Have multiple alternative solutions and keep close contact with customer & Detail what was attempted as well as why it couldn't be solved in the final report.\\
\hline
	Personal matters & 8 & 5 & 40 & Not much preventative action can be taken & Keep in touch and stay updated. In case you still can do tasks, claim one and tell the others\\
\hline
\end{tabularx}
\end{table}

