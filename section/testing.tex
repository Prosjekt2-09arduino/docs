\chapter{Testing}
	Testing is used to accomplish quality of the software and to avoid defects.
	
	\section{Testing strategy}
		\subsection{Unit testing}
			For the Android development an Android testing API (Robotium) was used.
			Robotium is a powerful test framework that could be used for function, system and acceptance testing.
			These tests aim at writing small portions of code and checking the results against expected results.
			For example the connection could be tested to check if it is consistent when we go into settings from the main activity.\\

			\begin{lstlisting}
solo.clickOnActionBarItem(R.id.settings);

ConnectionState expectedState = BluetoothConnection
	.ConnectionState.STATE_CONNECTED;
	
ConnectionState actualState = BtArduinoService.getBtService()
	.getBluetoothConnection().getConnectionState();

assertEquals(expectedState, actualState);
			\end{lstlisting}

			\subsubsection{Requirements tested this way}
			FR04 - Validation of Arduino hardware and software

		\subsection{Integration testing}
			This test is based on combining different units and testing the interface between them. It is a phase after the unit testing and before the validation testing. The units that can be used in these tests must therefore have passed the unit tests.

			Because our system is composed in more than one process (stk500 protocol and the market application), they
			were tested in pairs and not all at once.

			Integration testing identifies problems that occur when we combine the units.

			A persistent bluetooth connection through the whole application was tested by checking if the service was providing a STATE\_CONNECTED at all times when we expected it to be connected. When the application was closed and reopened or if the phone was rebooted it should hold its state if possible (if the arduino was turned on and in range). 

			\subsubsection{Requirements tested this way}
			FR01 - Over the air installation
			FR03 - Example PUIs

		\subsection{Validation testing}
			Validation testing is used for validating that the system has met the customer's requirements. This was done by testing that the system is holding the requirement specification.


		\subsection{Usability test}
		As the intent of the product was to ease the use of and set up physical user interfaces, usability tests were be crucial in the testing aspects. The decision were made to perform a hallway test following tasks given to the test subjects. To ensure proper understanding of the test subject, the think aloud protocol was used, meaning the subject talked about their actions during the test. There were also expert reviews made during a presentation at the offices of SINTEF. The think aloud protocol was also used here.

		All test subjects filled out a System Usability Scale form in order to create a score, 0-100, which gave us a comparable score with other systems. Concerning the amount of subjects tested, five users has been recognised as the proper amount. According to \cite{Nielsen}, 85\% of the problems with a system will be found by these five users, thus testing more would increase the time spent watching the same problems resurface.

		\subsubsection{Tasks}
		These were the tasks given to each tester. The tasks were created with the user cases previously mentioned as a basis. Each task is meant to be a vague description of what they are supposed to do in order for them having to think mostly for themselves and not given any answer as to what they are supposed to do. 

		The preface of these tasks were that they had bought a jacket with an Arduino device fitted in it. This jacket was called an iJacket and they had just downloaded the $\mu$CSS in order for them to check what other applications they could have on this jacket.

		\begin{enumerate}
		 \item Select your jacket
		 \item Look for applications
		 \item Filtrate the applications which doesn't work on your jacket
		 \item Install the application
		\end{enumerate}

		After the tests the test subjects were asked if there were any details or points during the test where they either were confused or found it difficult to do something. These points would be added or confirmed on a form where the observer would have noted any other details which may have risen during the test.

			\subsubsection{Requirements tested this way}
			FR02 - Easy to use interface

	\section{Test Results}
		\subsection{Unit Testing}
		All written Robotium tests run without encoutering errors. This indicates that the units tested works properly and correct values are obtained throughout the testing. A decision was made to not write any more tests as time was of an essence and resources were needed elsewhere. 

		\subsection{Integration Testing}

		\subsection{Validation Testing}
		What the customer thought 

		\subsection{Usability Testing}


%Add table with all requirements and whether they pass the tests or not

%FR01 Over the air installation
%		& The Android application and the Arduino device should communicate over Bluetooth$\textsuperscript{\textregistered}$  and install an arbitrary application from the Arduino Store in a simple two step process

%FR02  Easy to use interface
%& The Arduino Store application should be easy to use and easy to understand. It should not be necessary to do anything on the Arduino except for starting it. On startup it should search for nearby Bluetooth$\textsuperscript{\textregistered}$  connections with paired devices.\\

%FR03 Example PUIs
%& To demonstrate the Arduino Store (on Android), over the air installation, and the application in action on an Arduino.\\

%FR04 Validation of Arduino hardware and software
%& The Android application should by default hide Arduino applications in the Arduino Store which are incompatible with the Arduino device depending on memory requirements and connected devices.\\

