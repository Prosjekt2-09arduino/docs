\chapter{Testing}
	This chapter contains a description of the testing performed during the project. Tests were performed in order to ensure product quality and that the product met the requirements set.
	
	\section{Testing strategy}
		The ways in which testing was performed is detailed in this section. Testing was performed by unit testing first, then integration testing. Usability testing was performed after the main shop part of the project was completed. User acceptance testing was then performed by the customer after significant parts were complete.

		\subsection{Unit testing}
			For testing of the Android application, Robotium was used. Robotium is a powerful test framework that can be used for function, system and acceptance testing. The tests created in Robotium aims at writing small portions of code and checking the results against expected predefined values. Robotium was used for unit testing on the Android application. During the development of the application, different units were tested with Robotium after they were finished. An example of a Robotium test is shown below. In this test it is asserted that the Bluetooth connection with a Arduino device is as expected. \\

			\begin{lstlisting}
solo.clickOnActionBarItem(R.id.settings);

ConnectionState expectedState = BluetoothConnection
	.ConnectionState.STATE_CONNECTED;
	
ConnectionState actualState = BtArduinoService.getBtService()
	.getBluetoothConnection().getConnectionState();

assertEquals(expectedState, actualState);
			\end{lstlisting}

		\subsection{Integration testing}
		Integration testing is based on combining different units and testing the interface and communication between them. This type of testing comes after the unit testing phase and before the validation testing. The units that can be used in these tests must therefore have passed the unit tests. Because the system is composed of more than one component (STK500 protocol and the Android application), they were tested in pairs and not all at once. Integration testing identifies problems that occur when we combine the different units of the product, and was performed when units of the product was put together.

		\subsection{Usability test}
		As the intent of the product was to ease the use of and set up physical user interfaces, usability tests were crucial in the testing aspects. The decision were made to perform a hallway test following tasks given to the test subjects. To ensure that the test subjects had a proper understanding of the product, the think aloud protocol was used, meaning the subject talked about their actions during the test. \\
		\newline
		All test subjects filled out a System Usability Scale (SUS) form in order to create a score, 0-100, which gave us a comparable score with other systems. Concerning the amount of subjects tested, five users has been recognized as the proper amount. According to \cite{Nielsen}, 85\% of the problems with a system will be found by these five users, thus testing more would increase the time spent watching the same problems resurface.

		\subsubsection{Tasks}
			These were the tasks given to each tester. The tasks were created with the user cases previously mentioned as a basis. Each task is meant to be a vague description of what they are supposed to do in order for them having to think mostly for themselves and not given any answer as to what they are supposed to do.

			The preface of these tasks was that they had bought a jacket with an Arduino device fitted in it. This jacket was called an iJacket and they had just downloaded the $\mu$CSS in order for them to check what other applications they could have on this jacket.

			\vspace{6 mm}
			\begin{enumerate}
			 \item Select your jacket
			 \item Look for applications
			 \item Filtrate the applications which doesn't work on your jacket
			 \item Install the application
			\end{enumerate}
			\vspace{6 mm}

			After the tests the test subjects were asked if there were any details or points during the test where they either were confused or found it difficult to do something. These points would be added or confirmed on a form where the observer would have noted any other details which may have risen during the test.

		\subsection{User acceptance testing}
		User acceptance testing was performed with the customer in order to ensure that requirements were met and to check whether or not the customer was satisfied. This test was done in much the same way as the usability test, though with a scope bigger than just the usability aspect. A phone with the application was given to the customer in order for him to get first degree experience with the application. The tester was then asked to perform the same cases as in the usability tests, while thinking aloud in order to properly record his reactions to the application. These reactions were then written down on paper. After the cases were done with, the customer was given the opportunity to just play around with the product. Lastly, the customer was asked to what degree he felt the product fulfilled the requirements set for the product.\\
		\newline
		There were also expert reviews made during a presentation at the offices of SINTEF. The think aloud protocol was also used here. These testers had much experience within development, but not much knowledge about the task. They were therefore given a brief presentation beforehand about the project and given a short presentation of the application itself.\\


	\section{System Testing}
	\label{systemtesting}
	In this section the system tests will be presented. Each test was written to test a specific part of the product, and applies to both the Android application, STK500 protocol and software installed on the Arduino device. Each test was executed separately, and the results will be presented in section \ref{testresults}.

	\begin{table}[H]
	\caption{Connect with device using device list}
	\begin{tabularx}\linewidth{|m{0.15 \textwidth}|X|}
		\hline
			{Test ID} & {ST01}\\
		\hline
			Test name & Device list connect\\
		\hline
			Test description & Connect with an Arduino device from the Device list screen \\
		\hline
			Precondition & Arduino device is switched on with Bluetooth \\
		\hline
			Test steps & \begin{itemize}
				\item{Start program}
				\item{Press Device list from Action overflow or from the Welcome screen}
				\item{Press desired Arduino device}
				\end{itemize} \\
		\hline
			Success condition & Android and Arduino device is connected via Bluetooth \\
		\hline
	\end{tabularx}
	\end{table}

	\begin{table}[H]
	\caption{Connect with device using QR code}
	\begin{tabularx}\linewidth{|m{0.15 \textwidth}|X|}
		\hline
			{Test ID} & {ST02}\\
		\hline
			Test name & QR code connect\\
		\hline
			Test description & Connect with an Arduino device using QR code reader \\
		\hline
			Precondition & Arduino device is switched on with Bluetooth \\
		\hline
			Test steps & \begin{itemize}
				\item{Start program}
				\item{Press Device list from Action overflow}
				\item{Press Add device}
				\item{Press Connect with QR code}
				\item{Choose QR code reader}
				\item{Scan QR code}
				\end{itemize} \\
		\hline
			Success condition & Android and Arduino device is connected via Bluetooth \\
		\hline
	\end{tabularx}
	\end{table}

	\begin{table}[H]
	\caption{Connect with device using serial}
	\begin{tabularx}\linewidth{|m{0.15 \textwidth}|X|}
		\hline
			{Test ID} & {ST03}\\
		\hline
			Test name & Input serial connect \\
		\hline
			Test description & Connect with an Arduino device using serial \\
		\hline
			Precondition & Arduino device is switched on with Bluetooth \\
		\hline
			Test steps & \begin{itemize}
				\item{Start program}
				\item{Press Device list from Action overflow}
				\item{Press Add device}
				\item{Press Connect with QR code}
				\item{Choose QR code reader}
				\item{Scan QR code}
				\end{itemize} \\
		\hline
			Success condition & Android and Arduino device is connected via Bluetooth \\
		\hline
	\end{tabularx}
	\end{table}

	\begin{table}[H]
	\caption{Search for desired app}
	\begin{tabularx}\linewidth{|m{0.15 \textwidth}|X|}
		\hline
			{Test ID} & {ST04}\\
		\hline
			Test name & Search \\
		\hline
			Test description & Search for a specific app \\
		\hline
			Precondition & Database is populated \\
		\hline
			Test steps & \begin{itemize}
				\item{Start program}
				\item{Press search icon}
				\item{Type search string}
				\end{itemize} \\
		\hline
			Success condition & Search result is shown with matching apps \\
		\hline
	\end{tabularx}
	\end{table}

	\begin{table}[H]
	\caption{Connect to last connected device when in range}
	\begin{tabularx}\linewidth{|m{0.15 \textwidth}|X|}
		\hline
			{Test ID} & {ST05}\\
		\hline
			Test name & Last connected\\
		\hline
			Test description & Test that the Android application connects to the last connected Arduino device when it is in range and gives proper feedback to the user. The application should only reconnect when this option is selected. \\
		\hline
			Precondition & Available Arudino device with Bluetooth \\
		\hline
			Test steps & \begin{itemize}
				\item{Check option ''Automatically try to reconnect to last connected device''}
				\item{Connect with an Arduino}
				\item{Exit Android application completely}
				\item{Start Android application}
					\begin{itemize}
						\item{When the Arduino device is in range}
						\item{When the Arduino device is out of range/off}
					\end{itemize}
				\end{itemize} \\
		\hline
			Success condition & When the last connected Arudino device is on and in range, the Android device should connect to it immediately when the Android application is started and give feedback to the user. If the Arduino device is out of range or off, proper feedback should be given \\
		\hline
	\end{tabularx}
	\end{table}

	\begin{table}[H]
	\caption{Browse apps test}
	\begin{tabularx}\linewidth{|m{0.15 \textwidth}|X|}
		\hline
			Test ID & ST06\\
		\hline
			Test name & Browse apps\\
		\hline
			Test description & Browse apps in the Android application in a random fashion \\
		\hline
			Precondition & Database is populated \\
		\hline
			Test steps & \begin{itemize}
				\item{Start program}
				\item{Choose a category}
				\item{Choose an app}
				\item{Go back and choose different app}
				\item{Swipe left/right to browse different sorting}
				\end{itemize} \\
		\hline
			Success condition & User is able to browse apps \\
		\hline
	\end{tabularx}
	\end{table}

	\begin{table}[H]
	\caption{Install application on Arduino device}
	\begin{tabularx}\linewidth{|m{0.15 \textwidth}|X|}
		\hline
			{Test ID} & {ST07}\\
		\hline
			Test name & Install application on Arduino device\\
		\hline
			Test description & Choose a desired application to install on a connected Arduino device \\
		\hline
			Precondition & Android and Arduino device is connected via Bluetooth \\
		\hline
			Test steps & \begin{itemize}
				\item{Start program}
				\item{Select desired app}
				\item{Press Install}
				\item{Press Confirm}
				\end{itemize} \\
		\hline
			Success condition & Arduino app is installed on the connected Arduino device \\
		\hline
	\end{tabularx}
	\end{table}

	\begin{table}[H]
	\caption{Change connected device}
	\begin{tabularx}\linewidth{|m{0.15 \textwidth}|X|}
		\hline
			{Test ID} & {ST08}\\
		\hline
			Test name & Change connected device\\
		\hline
			Test description & Test that it is possible to first connect to one Arduino, then another, thus changing which device the Android application is connected to \\
		\hline
			Precondition & Two available Arduinos with Bluetooth\\
		\hline
			Test steps & \begin{itemize}
				\item{Start program}
				\item{Connect with Arduino \#1}
				\item{Connect with Arduino \#2}
				\end{itemize} \\
		\hline
			Success condition & Both connections is successful, and the Android device is connected with Arduino \#2 \\
		\hline
	\end{tabularx}
	\end{table}

	\section{Test Results}
	\label{testresults}
		In this section the results of all the tests performed will be presented. In each subsection the result of the given type of test will be shown.

		\subsection{Unit Testing}
		All written Robotium tests run without encountering errors. This indicates that the units tested works properly and correct values are obtained throughout the testing. A decision was made to not write any more tests as time was of an essence and resources were needed elsewhere.

		\subsection{Integration Testing}
		When different units of the product was ready to be sown together, integration testing was performed. When bugs and errors was encountered, the person currently responsible for the integration testing would fix it. %Write more here?

		\subsection{System testing}
		The system testing was performed according to the description of the tests in section \ref{systemtesting}. In table~\ref{table:systemtestingtable} the results of each test is shown. These tests was performed when both unit testing an integration testing was done to ensure that the product met the requirements set by the customer.

		\begin{table}[H]
		\caption{System test results}
		\label{table:systemtestingtable}
		\begin{tabularx}\linewidth{|m{0.15 \textwidth}|m{0.15 \textwidth}|X|}
			\hline
				{\bf Test ID} & {\bf Result} & {\bf Comment}\\
			\hline
				ST01 & Passed & If the Arduino device is on and within range, the Android application successfully connects with it via Bluetooth using the Device List. \\
			\hline
				ST02 & Passed & If the Arduino device is on and within range, and the QR code is correct, the Android application successfully connects with it via Bluetooth using optional QR reader. \\
			\hline
				ST03 & Passed & If the Arduino device is on and within range, the Android application successfully connects with it via Bluetooth by providing the correct MAC address of the device to the Android application. \\
			\hline
				ST04 & Failed & The application is crashing when a search is performed. \\
			\hline
				ST05 & Passed & The application automatically tries to reconnect to the last connected device when this option is selected. It gives proper feedback to the user containing the result of the connection attempt. \\
			\hline
				ST06 & Passed & It is possible to browse apps in the Android application in a random fashion without problems. \\
			\hline
				ST07 & PENDING & This test has not yet been performed. \\
			\hline
				ST08 & Passed & It is possible to change the connected device without disconnecting first.\\
			\hline
		\end{tabularx}
		\end{table}

		\subsection{Usability Testing}
		Usability testing has not yet been performed.

		\subsection{User Acceptance Testing}
		After the presentation by the customers request at the SINTEF offices, three people agreed to fill out SUS forms about their thoughts of the system. As there were still problems with the system, the scores were not too positive, but received an average score of 54 of a maximum of 100. The main ideas for improvements were stability and security. Stability was then worked on, but the security aspect was not recognized as an important focus point this late in the development. This was agreed upon by the customer after the meeting.

		\subsection{Summary}
		Based on the test performed during and after development of the product, table x describes to what degree the initial requirements set by the customer were met. As described in the table \ref{table:functionalsummary}, some of the requirements were omitted during the project as some tasks proved more time consuming than expected. As a result of this, not all the initial requirements have been met. \\
		\newline
		Regarding the non-functional requirements, it is difficult to summarize to what degree some of these requirements have been met. As a result of this not all the non-functional requirements is included in the table below.

		\begin{table}[H]
		\caption{Functional requirements}
		\label{table:functionalsummary}
		\begin{tabularx}\linewidth{|m{0.15 \textwidth}|m{0.15 \textwidth}|X|}
			\hline
				{\bf Requirement code} & {\bf Result} & {\bf Comment}\\
			\hline
				FR01 & PENDING & This cannot be tested before the protocol is complete\\
			\hline
				FR02 & PENDING & Usability testing will be performed at SINTEF workshop later\\
			\hline
				FR03 & PASS & Two example PUIs were made to prove the functionality of the product \\
			\hline
				FR04 & PASS & A valid Bluetooth connection stays active even if the Android application is minimized. When the application is killed and restarted, it will connect to the last connected Arduino device if in range\\
			\hline
				FR05 & FAIL & In agreement with the customer, this requirements was omitted \\
			\hline
				FR06 & PASS & The user of the Android application can connect with an Arduino using QR code reader, input serial or choose from a list of available Arduino devices\\
			\hline
				NFR02 & PASS & The Android application is stable, and the Bluetooth connection is persistent even throughout the program \\
			\hline
				NFR03 & PASS & All code written is open source and under Apache 2.0 licence \\
			\hline
				NFR04 & PASS & The Android application is at least compatible with Android 2.3 and newer \\
			\hline
		\end{tabularx}
		\end{table}

