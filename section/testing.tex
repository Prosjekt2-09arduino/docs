\chapter{Testing}
	This chapter contains a description of the testing performed during the project. Tests were performed in order to ensure product quality and that the product meets the requirements set. 
	
	\section{Testing strategy}
		The ways in which testing was performed is detailed in this section. Testing was performed by unit testing first, then integration testing. Usability testing was performed after the main shop part of the project was completed. User acceptance testing was then performed by the customer after significant parts were complete.

		\subsection{Unit testing}
			For testing of the Android application, Robotium was used. Robotium is a powerful test framework that can be used for function, system and acceptance testing. The tests created in Robotium aims at writing small portions of code and checking the results against expected predefined values. Robotium was used for unit testing on the Android application. During the development of the application, different units were tested with Robotium after they were finished. An example of a Robotium test is shown below. In this test it is asserted that the Bluetooth connection with a Arduino device is as expected. \\

			\begin{lstlisting}
solo.clickOnActionBarItem(R.id.settings);

ConnectionState expectedState = BluetoothConnection
	.ConnectionState.STATE_CONNECTED;
	
ConnectionState actualState = BtArduinoService.getBtService()
	.getBluetoothConnection().getConnectionState();

assertEquals(expectedState, actualState);
			\end{lstlisting}

		\subsection{Integration testing}
			Integration testing is based on combining different units and testing the interface and communication between them. It is a phase after the unit testing and before the validation testing. The units that can be used in these tests must therefore have passed the unit tests. Because our system is composed of more than one process (stk500 protocol and the market application), they were tested in pairs and not all at once. Integration testing identifies problems that occur when we combine the different units of the product. A persistent Bluetooth connection through the whole application was tested by checking if the service was providing a STATE\_CONNECTED at all times when we expected it to be connected. When the application was closed and reopened or if the phone was rebooted it should hold its state if possible (if the Arduino was turned on and in range). 


		\subsection{Usability test}
		As the intent of the product was to ease the use of and set up physical user interfaces, usability tests were crucial in the testing aspects. The decision were made to perform a hallway test following tasks given to the test subjects. To ensure proper understanding of the test subject, the think aloud protocol was used, meaning the subject talked about their actions during the test. 

		All test subjects filled out a System Usability Scale form in order to create a score, 0-100, which gave us a comparable score with other systems. Concerning the amount of subjects tested, five users has been recognised as the proper amount. According to \cite{Nielsen}, 85\% of the problems with a system will be found by these five users, thus testing more would increase the time spent watching the same problems resurface.

		\subsubsection{Tasks}
		These were the tasks given to each tester. The tasks were created with the user cases previously mentioned as a basis. Each task is meant to be a vague description of what they are supposed to do in order for them having to think mostly for themselves and not given any answer as to what they are supposed to do. 

		The preface of these tasks were that they had bought a jacket with an Arduino device fitted in it. This jacket was called an iJacket and they had just downloaded the $\mu$CSS in order for them to check what other applications they could have on this jacket.

		\begin{enumerate}
		 \item Select your jacket
		 \item Look for applications
		 \item Filtrate the applications which doesn't work on your jacket
		 \item Install the application
		\end{enumerate}

		After the tests the test subjects were asked if there were any details or points during the test where they either were confused or found it difficult to do something. These points would be added or confirmed on a form where the observer would have noted any other details which may have risen during the test.


		\subsection{User acceptance testing}
			User acceptance testing was done by the customer in order to ensure that requirements were met and whether the customer had any improvement points or not. There were also expert reviews made during a presentation at the offices of SINTEF. The think aloud protocol was also used here.


	\section{System Testing}
	Write about system testing here.

	\section{Test Results}
		In this section the results of all the tests performed will be presented. In each subsection the result of the given type of test will be shown.

		Table ~\ref{table:testresults} shows the test results of the Robotium Unit tests

		\begin{table}[ht!]
		\caption{Test Results}
		\label{table:testresults}
		\begin{tabularx}\linewidth{|m{0.15 \textwidth}|m{0.15 \textwidth}|X|}
			\hline
				{\bf Requirement code} & {\bf Result} & {\bf Comment}\\
			\hline
				FR01 & PENDING & This cannot be tested before the protocol is complete\\
			\hline
				FR02 & PASS & Usability testing was performed on SINTEF workshop\\
			\hline
				FR03 & PASS & \\
			\hline
				FR04 & PASS & This is possible unless you uninstall and install the application\\
			\hline
				FR05 & ERROR & There was no time to implement a standard for device compatibility validation\\
			\hline
				FR06 & PASS & The user can choose from a list a QR reader to use. The serialInput is a mac address\\
			\hline
		\end{tabularx}
		\end{table}

		\subsection{Unit Testing}
		All written Robotium tests run without encoutering errors. This indicates that the units tested works properly and correct values are obtained throughout the testing. A decision was made to not write any more tests as time was of an essence and resources were needed elsewhere. 

		\subsection{Integration Testing}

		\subsection{Usability Testing}

		\subsection{User Acceptance Testing}
		What the customer thought 
