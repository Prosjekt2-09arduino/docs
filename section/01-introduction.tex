\section{Introduction}
\subsection{The group}
The group consisted of six students from Computer Science at NTNU, each possessing different technical skills.

\subsection{Problem description}
The project task was to develop a platform for easy use, setup and sharing of Physical User Interfaces (PUIs) for Arduino. The task was divided in three main parts: a market application, over the air installation and example PUIs.\\
\newline
The purpose of the market application was to allow users of Arduino to browse and download PUI applications for their Arduino board on a mobile device. This market app was to be a simplified version of e.g. Google Play, where users easily can browse and install whatever applications they desire. An application selected for installation in the market application should be prepared on the mobile device, and installed over the air on an Arduino board using Bluetooth. The example PUIs were mostly intended to demonstrate the feasibility of the finished product.

\subsection{The goal}
The goal of the project task was to make Arduino easier to use for ordinary people, by allowing easy browsing, sharing and installation of applications on Arduino boards. By developing an application for over the air installation of applications, the finished product should ease the process of both installing and updating PUIs on an Arduino.

\subsection{Work breakdown structure (WBS)}
The WBS is a view on what work packages the project encompasses. It helps with communicating the work and processes to easily execute the project. The duration-time show how much estimated time one task require, and gives an assessment on how much effort to be considered.\\
\newline
\textbf{The Dictionary View} is an organized table view of the WBS with a simple structure. The Duration column is an extension of the original Dictionary View.\\

\includepdf[pages={-}]{figures/wbs.pdf}

\textbf{The Tree Structure}, which is the most popular WBS format, presents an easy to understand view into the WBS. The Tree Structure is not easy to make and is created with a tool: Microsoft Word and SmartArt graphics.\\
\begin{figure}[H]
\includegraphics[scale=0.7]{figures/wbstree.pdf}
\captionof{figure}{WBS Tree}
%\label{fig:WBStree}
\end{figure}